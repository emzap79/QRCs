% Reference Card for AMS-LaTeX
% Copyright (c) 2000 Joseph H. Silverman. May be freely distributed.
% Created Friday, December 24, 1999
% Thanks to Stephen Gildea for the multicolumn macro package
% which I modified from his GNU emacs reference card

%**start of header
\newcount\columnsperpage

% This file can be printed with 1, 2, or 3 columns per page (see below).
% [For 2 or 3 columns, you'll need 6 and 8 point fonts.]
% Specify how many you want here.  Nothing else needs to be changed.

\columnsperpage=2

% Smaller (97%) pdf file with horizontal offset 1.5in
% dvipdfm -l -m 0.97 -x 1.5in -o LaTeXRefCard.v2.0.pdf LaTeXRefCard.v2.0.dvi

% This file is intended to be processed by plain TeX.
%
% The final reference card has six columns, three on each side.
% This file can be used to produce it in any of three ways:
% 1 column per page
%    produces six separate pages, each of which needs to be reduced to 80%.
%    This gives the best resolution.
% 2 columns per page
%    produces three already-reduced pages.
%    You will still need to cut and paste.
% 3 columns per page
%    produces two pages which must be printed sideways to make a
%    ready-to-use 8.5 x 11 inch reference card.
%    For this you need a dvi device driver that can print sideways.
% Which mode to use is controlled by setting \columnsperpage above.
%
% Author:
%  Joseph H. Silverman
%  Brown University Mathematics Department
%  Providence, RI 02912 USA
%  Internet:  jhs@math.brown.edu
%  (reference card macros due to Stephen Gildea)

% History:
%  Version 0.9 - January 2000, first distribution
%  Version 1.0 - Janaury 2003, corrected version for general distribution
%  Version 2.0 - January 2007, deleted slide section, deleted letter section
%                added picture section, added color section, 
%                added a few command, made minor corrections

\def\versionnumber{2.0}  % Version of this reference card
\def\year{2007}
\def\month{January}
\def\version{\month\ \year\ v\versionnumber}


\def\shortcopyrightnotice{
\vskip 0pt plus 2 fill\begingroup\parskip=0pt\small
   \centerline{\copyright\ \number\year\ J.H. Silverman,
   Permissions on back.  v\versionnumber}

Send comments and corrections to J.H. Silverman, Math.\ Dept.,
Brown Univ., Providence, RI 02912 USA.
$\langle$jhs@math.brown.edu$\rangle$

\endgroup}

\def\copyrightnotice{
\vskip 1ex plus 2 fill\begingroup\parskip=0pt\small
\centerline{Copyright \copyright\ \year\ J.H. Silverman, \version}
\centerline{Math.\ Dept., Brown Univ., Providence, RI 02912 USA}

Permission is granted for noncommercial distribution 
%% of this card 
provided the copyright notice and this permission notice
are preserved on all copies.

\endgroup}

% make \bye not \outer so that the \def\bye in the \else clause below
% can be scanned without complaint.
\def\bye{\par\vfill\supereject\end}

\newdimen\intercolumnskip
\newbox\columna
\newbox\columnb

\def\ncolumns{\the\columnsperpage}

\message{[\ncolumns\space
   column\if 1\ncolumns\else s\fi\space per page]}

\def\scaledmag#1{ scaled \magstep #1}

% This multi-way format was designed by Stephen Gildea
% October 1986.
\if 1\ncolumns
   \hsize 4in
   \vsize 10in
   \voffset -.7in
   \font\titlefont=\fontname\tenbf \scaledmag3
   \font\headingfont=\fontname\tenbf \scaledmag2
   \font\smallfont=\fontname\sevenrm
   \font\smallsy=\fontname\sevensy
   \font\tenss=cmss10
   \def\ss{\tenss}
   \footline{\hss\folio}
   \def\makefootline{\baselineskip10pt\hsize6.5in\line{\the\footline}}
\else
   \hsize 3.2in
   \vsize 7.95in
   \hoffset -.75in
   \voffset -.745in
   \font\titlefont=cmbx10 \scaledmag2
   \font\headingfont=cmbx10 \scaledmag1
   \font\headingfonttt=cmtt10 \scaledmag1
   \font\smallfont=cmr6
   \font\smallsy=cmsy6
   \font\eightrm=cmr8
   \font\eighti=cmmi8
   \font\eightsy=cmsy8
   \font\eightbf=cmbx8
   \font\eighttt=cmtt8
   \font\eightit=cmti8
   \font\eightsl=cmsl8
   \font\eightss=cmss8
   \textfont0=\eightrm
   \textfont1=\eighti
   \textfont2=\eightsy
   \def\rm{\eightrm}
   \def\bf{\eightbf}
   \def\tt{\eighttt}
   \def\it{\eightit}
   \def\sl{\eightsl}
   \def\ss{\eightss}
   \normalbaselineskip=.8\normalbaselineskip
   \normallineskip=.8\normallineskip
   \normallineskiplimit=.8\normallineskiplimit
   \normalbaselines\rm          %make definitions take effect

   \if 2\ncolumns
     \let\maxcolumn=b
     \footline{\hss\rm\folio\hss}
     \def\makefootline{\vskip 2in \hsize=6.86in\line{\the\footline}}
   \else \if 3\ncolumns
     \let\maxcolumn=c
     \nopagenumbers
   \else
     \errhelp{You must set \columnsperpage equal to 1, 2, or 3.}
     \errmessage{Illegal number of columns per page}
   \fi\fi

%  \intercolumnskip=.46in
   \intercolumnskip=.36in
   \def\abc{a}
   \output={%
       % This next line is useful when designing the layout.
       %\immediate\write16{Column \folio\abc\space starts with \firstmark}
       \if \maxcolumn\abc \multicolumnformat \global\def\abc{a}
       \else\if a\abc
        \global\setbox\columna\columnbox \global\def\abc{b}
         %% in case we never use \columnb (two-column mode)
         \global\setbox\columnb\hbox to -\intercolumnskip{}
       \else
        \global\setbox\columnb\columnbox \global\def\abc{c}\fi\fi}
   \def\multicolumnformat{\shipout\vbox{\makeheadline
       \hbox{\box\columna\hskip\intercolumnskip
         \box\columnb\hskip\intercolumnskip\columnbox}
       \makefootline}\advancepageno}
   \def\columnbox{\leftline{\pagebody}}

   \def\bye{\par\vfill\supereject
     \if a\abc \else\null\vfill\eject\fi
     \if a\abc \else\null\vfill\eject\fi
     \end}
\fi

\def\SPC{\quad} % space between symbol and command

\parindent 0pt
\parskip 1ex plus .5ex minus .5ex

\def\small{\smallfont\textfont2=\smallsy\baselineskip=.8\baselineskip}

\outer\def\newcolumn{\vfill\eject}

\outer\def\title#1{{\titlefont\centerline{#1}}\vskip 1ex plus .5ex minus.5ex}

\outer\def\section#1{%
   % next line is useful for organizing placement of sections
   %\immediate\write16{Column \folio\abc\space: #1}%
   \par\filbreak
   \vskip 1ex plus 2ex minus 2ex {\headingfont #1}\mark{#1}%
   \vskip 1ex plus 1ex minus .5ex}


\def\paralign{\vskip\parskip\halign}

\def\begintext{\par\leavevmode\begingroup\parskip0pt\rm}
\def\endtext{\endgroup}

%%%%%%% Boxes Around Formulas %%%%%%%%%%%%%%%%
\def\boxit#1{{\setbox0=\hbox{\kern2pt#1\kern2pt}
    \dimen0=\ht0 \advance\dimen0 by1.5pt
    \dimen1=\ht0 \advance\dimen1 by1.9pt
    \dimen2=\dp0 \advance\dimen2 by1.5pt
    \dimen3=\dp0 \advance\dimen3 by1.9pt
    \vrule height\dimen1 depth-\dimen0 width\wd0  \kern-\wd0
    \vrule height-\dimen2 depth\dimen3 width\wd0  \kern-\wd0
    \vrule height\dimen1 depth\dimen3 width.4pt  \kern\wd0\kern-.8pt
    \vrule height\dimen1 depth\dimen3 width.4pt  \kern-\wd0
    \box0\thinspace}}

%%%%% AMS-LaTeX Logo %%%%%%%%%%%%
\font\smalltitle=cmbx7 \scaledmag2
\font\titlecal=cmsy10 \scaledmag2
\def\LaTeX{L\kern-5.6pt\raise4pt\hbox{\smalltitle A}\kern-1.6pt\TeX}
\def\AMS{{\titlecal\char"41\kern-3.2pt\lower4pt\hbox{\char"4D}\kern-1pt\char"53}}
\def\AMSLaTeX{\AMS\raise1.5pt\hbox{-}\LaTeX}
\def\smallLaTeX{L\kern-3pt\raise2pt\hbox{\smash{\sevenrm A}}\kern-1pt\TeX}

\chardef\\=`\\
\chardef\{=`\{
\chardef\}=`\}
\chardef\_=`\_

\font\eightmsb=msbm8
\newfam\msbfam
\textfont\msbfam=\eightmsb
\def\Bbb#1{{\fam\msbfam\relax#1}}
\def\nmid{{\eightmsb\char"2D}}
\font\eighteufm=eufm8
\newfam\eufmfam
\textfont\eufmfam=\eighteufm
\def\frak#1{{\fam\eufmfam\relax#1}}
\font\eighteusm=eusm8
\newfam\eusmfam
\textfont\eusmfam=\eighteusm
\def\script#1{{\fam\eusmfam\relax#1}}

\def\cs#1{{\tt\\#1}}
\def\css#1#2{{\tt\\#1 \dots\\#2}}
\def\qcs#1{\quad{\tt\\#1}}
\def\qcss#1#2{\quad{\tt\\#1 \dots\\#2}}
\def\qqcs#1{\qquad{\tt\\#1}}
\def\<#1>{$\langle${\rm #1}$\rangle$}
\def\HR{\noalign{\hrule height.4pt depth-.35pt width 1in}}
\def\hcr{\hidewidth\cr}

%%%%%%%%%%%%%%%%%  TEXT STARTS HERE %%%%%%%%%%%%%%%%
\title{\AMSLaTeX~Reference Card \#1}
\smallskip
\centerline{See the \TeX\ Reference Card for additional commands.}
\centerline{Required packages are indicated as {(\ss package)}.}

\section{Document Structure}
\halign{&#\hfil\quad\cr
$\bullet$ {\bf Preamble}\hcr
   &\cs{documentclass[{\rm option(s)}]\{{\rm class}\}}\hcr
   &\cs{usepackage[{\rm option(s)}]\{{\rm package(s)}\}}\hcr
\cs{begin\{document\}}\hcr
$\bullet$ {\bf Body}\hcr
   &$\bullet$ {\bf Front Matter} \quad(\cs{frontmatter}
         in {\tt book} classes)\hcr
     &&$\bullet$ {\bf Top Matter}\hcr
       &&&\cs{title\{\dots\}}\hcr
       &&&\cs{title[{\rm running head}]\{\dots\}} alternative headline\cr
       &&&\cs{date\{\dots\}}\hcr
       &&&\cs{date\{\cs{today}\}} gives current date\hcr
       &&&\cs{author\{\dots\}}\hcr
       &&&\cs{maketitle}\quad(not in {\tt book} classes)\hcr
%
       &&&$\bullet$ {\bf Additional items --- {\ss ams} classes only}\hcr
%
       &&&\cs{translator\{\dots\}}\hcr
       &&&\cs{dedicatory\{\dots\}}\hcr
       &&&\cs{address[{\rm optional name}]\{\dots\}}\hcr
       &&&\cs{curraddress\{\dots\}}\hcr
       &&&\cs{email[{\rm optional name}]\{\dots\}}\hcr
       &&&\cs{thanks\{\dots\}}\hcr
       &&&\cs{subjclass\{Primary:\thinspace XXX;
          Secondary:\thinspace XXX\}}\hcr
       &&&\cs{keywords\{\dots\}}\hcr
       &&&\cs{thanks\{\dots\}}\hcr
     &&\cs{tableofcontents}\hcr
     &&\cs{chapter\{Introduction\}}\quad (in {\tt book} classes)\hcr
     &&$\bullet${\bf Abstract}\quad(not in {\tt book} classes)\hcr
       &&&\cs{begin\{abstract\}}%
            \dots\cs{end\{abstract\}}\hcr
   &$\bullet$ {\bf Main Matter} \quad(\cs{mainmatter}
         in {\tt book} classes)\hcr
     &&\cs{chapter\{\dots\}}\hcr
     &&\cs{section\{\dots\}}\hcr
     &&\cs{subsection\{\dots\}}\hcr
     &&\cs{appendix}\hcr
   &$\bullet$ {\bf Back Matter} \quad(\cs{backmatter}
         in {\tt book} classes)\hcr
     &&\cs{begin\{thebibliography\}\{99\}}%
         \dots\cs{end\{\dots\}}\hcr
\cs{end\{document\}}\hcr
}

\section{Page Style}
\halign{#\hfil\quad&#\hfil\cr
\cs{pagestyle\{{\rm style}\}}&set page style to one of:\cr
\quad\tt plain&empty header, page number in footer\cr
\quad\tt empty&empty header and footer\cr
\quad\tt headings&header filled by doc class, empty footer\cr
\quad\tt myheadings&empty footer, fill header with info in\cr
     &\qquad\cs{markboth\{{\rm lefthead}\}\{{\rm righthead}\}}\cr
     &\qquad and \cs{markright\{{\rm righthead}\}}\cr
\multispan2%
\cs{thispagestyle\{{\rm style}\}}\hfill set \cs{pagestyle}, only 
current page\cr
\multispan2%
\cs{enlargethispage\{\cs{baselineskip}\}}
   \hfill force an extra line\cr
\multispan2%
\cs{renewcommand\{\cs{baselinestretch}\}\{2\}}
   \hfill doublespaced\cr
\multispan2%
{\ss fancyheadings}\hfill package allows custom headers and footers\cr
$\bullet$ {\bf Page Style Parameters}\hcr
\cs{hoffset}, \cs{voffset}\quad move page right, down\hcr
\cs{paperwidth}, \cs{paperheight}, \cs{textheight}, \cs{textwidth} 
\hcr
\cs{topmargin}, \cs{headheight}, \cs{headsep}, \cs{footskip}\hcr
\cs{pagenumbering\{\dots\}} \quad e.g., arabic, roman\hcr
}

\section{Classes and Packages}
\halign{&#\hfil\quad\cr
\cs{documentclass[{\rm option(s)}]\{{\rm class}\}}\hcr
\cs{usepackage[{\rm option(s)}]\{{\rm package(s)}\}}\hcr
\cs{NeedsTeXFormat\{LaTeX2e\}[1994/12/01]}\hcr
$\bullet${\bf Document Classes} \hcr
&\tt article, book, letter, report, slides \hcr
&{\tt amsart, amsbook, amsproc} (all autoload {\ss amsmath}) \hcr
$\bullet${\bf Useful Packages} \hcr
&&\tt amsmath,amsthm,amscd,amssymb,latexsym\hcr
&&{\tt fancyheadings}\quad allows custom headers and footers\hcr
&&{\tt alltt}\quad all teletype, even {\tt\\},{\tt\{},{\tt\}}\hcr
&&{\tt makeidx,showidx}
     \quad create index, show in margin\hcr
&&{\tt graphics,graphicx}\quad inclusion of graphics\hcr
&&{\tt enumerate}\quad extends the enumerate environment\hcr
&&{\tt layout\ \ }\quad shows page layout of doc class\hcr
&&{\tt multicol}\quad flexible multicolumn typesetting\hcr
&&{\tt showkeys}\quad print label keys in margin\hcr
&&{\tt verbatim}\quad extends verbatim environment\hcr
&&{\tt url\ \ \ \ \ }\quad typeset URLs allowing line breaks\hcr
&&{\tt graphpap}\quad \cs{graphpaper} command for \cs{picture} environ.\hcr
$\bullet${\bf Document and Package Options} \hcr
&Font Size \hcr
&&\tt 8pt, 9pt, 10pt, 11pt, 12pt \hcr
&Paper Size \hcr
&&\tt a4paper,a5paper,b5paper,legalpaper,letterpaper \hcr
&Document Preparation \hcr
&&\tt draft,final,notitlepage,titlepage \hcr
&Page Formatting \hcr
&&\tt onecolumn,twocolumn,oneside,twoside,openany,openright \hcr
&Equation Numbering \hcr
&&\tt fleqn,leqno,reqno,centertags,tbtags\hcr
&Equation Limits \hcr
&&\tt intlimits,sumlimits,nonamelimits\hcr
&AMS (Postscript) Fonts\hcr
&&\tt psamsfonts,noamsfonts\hcr
}

\section{Bibliography (see also B{\tenbf IB}\hglue-1pt\TeX)}
\halign{#\hfil\qquad&#\hfil\cr
\cs{begin\{thebibliography\}\{99\}\dots\cs{end}\{\dots\}}\hcr
\multispan2%
    \hfill bibliography with widest label specified\cr
\cs{bibitem\{{\rm name}\}}&named bibliography item\cr
\multispan2%
\cs{bibitem[{\rm label}]\{{\rm name}\}}
\hfill with alternative label to print\cr
\cs{bysame}&use long line for same author\cr
\multispan2%
\cs{renewcommand\{\cs{bibname}\}\{{\rm title}\}} \hfill
      use custom title\cr
\cs{cite\{{\rm name}\}}&print number of named bib item\cr
\cs{cite[{\rm text}]\{{\rm name}\}}&\qquad with extra text\cr
}

\section{Cross Referencing and Numbering}
\halign{#\hfil\quad&#\hfil\cr
\cs{label\{{\rm name}\}}&assign label name to numbered item\cr
\cs{ref\{{\rm name}\}}&print number of named item\cr
\cs{eqref\{{\rm name}\}}&print number in parentheses {\ss(amsmath)}\cr
\cs{pageref\{{\rm name}\}}&print page location of named item\cr
\cs{cite\{{\rm name}\}}&print number of named bibliography item\cr
\cs{cite[{\rm text}]\{{\rm name}\}}&\qquad with extra text\cr
\multispan2%
\cs{numberwithinsection\{equation\}\{section\}}\hfill number by section\cr
}

%% \copyrightnotice ???

\section{Sectioning and Table of Contents}
\halign{#\hfil\qquad&#\hfil\cr
$\bullet$ {\bf Sectioning commands}\hcr
\cs{command\{{title}\}}&sectioning command with title\cr
\cs{command[{\rm head}]\{{title}\}}&with alternative running head\cr
\cs{command*\{{title}\}}&with number supressed\cr
%
\noalign{\vskip1\jot\vbox{
\halign{&\qquad#\hfil\cr
\cs{part}&\cs{section}&\cs{paragraph}\cr
\cs{chapter}&\cs{subsection}&\cs{subparagraph}\cr
&\cs{subsubsection}\cr
}}}
%
\cs{appendix}&start appendix\cr
}
\halign{#\hfil\qquad&#\hfil\cr
$\bullet$ {\bf Table of Contents}\hcr
\cs{tableofcontents}&create and print contents\cr
\tt filename.toc&contents associated to {\tt filename.tex}\cr
\cs{addcontentsline\{toc\}\{section\}\{%
     {\rm line to add}\}}\hcr
\cs{addtocontents\{toc\}\{{\rm material to add}\}}\hcr
\cs{setcounter\{tocdepth\}\{\dots\}}\quad set amount to print\hcr
}

\section{Tables and Figures}
\halign{#\hfil\quad&#\hfil\cr
\cs{begin\{table\} \dots} \cs{caption\{}text{\tt\}}
   \cs{label\{{\rm name}\}} \cs{end\{table\}}\hcr
\cs{listoftables}&create and print list of tables\cr
\multispan2%
\cs{begin\{figure\} \dots} \cs{caption\{}text{\tt\}}
   \cs{label\{{\rm name}\}} \cs{end\{figure\}}\cr
\cs{includegraphics\{{\rm filename}\}}&include image {(\ss graphics)}\cr
\multispan2%
\cs{scaledbox\{.5\}\{\cs{includegraphics}\{{\rm filename}\}\}}
    \hfill scaled graphic\cr
\cs{listoffigures}&create and print list of figures\cr
}

\section{Lists}
\halign{#\hfil\quad&#\hfil\cr
\cs{item}&item within list\cr
\cs{item[{\rm label}]}&item with label\cr
\cs{begin\{enumerate\}\dots\cs{end}\{\dots\}}
   &numbered items\cr
\cs{begin\{itemize\}\dots\cs{end}\{\dots\}}
   &bulleted items\cr
\cs{begin\{description\}\dots\cs{end}\{\dots\}}
   &captioned items\cr
\cs{setlength\{\cs{itemsep}\}\{0pt\}}
   &move items closer\cr
{\ss enumerate} package&extends {\tt enumerate}\cr
}

\section{Displayed Text Material}
\halign{#\hfil\quad&#\hfil\cr
\cs{begin\{center\}\dots\cs{end}\{\dots\}}
   &centered matrial\cr
\cs{begin\{flushright\}\dots\cs{end}\{\dots\}}
   &flush right matrial\cr
\cs{begin\{flushleft\}\dots\cs{end}\{\dots\}}
   &flush left matrial\cr
\cs{begin\{quote\}\dots\cs{end}\{\dots\}}
   &short quote\cr
\cs{begin\{quotation\}\dots\cs{end}\{\dots\}}
   &long quote\cr
\cs{begin\{verse\}\dots\cs{end}\{\dots\}}
   &poetry\cr
\cs{begin\{verbatim\}\dots\cs{end}\{\dots\}}
   &verbatim material\cr
\cs{verb|\dots|}&verbatim material\cr
\multispan2%
\cs{verb*|\dots|}\hfill verbatim with spaces marked\cr
{\ss verbatim} package&extends {\tt verbatim}\cr
}

\section{Footnotes, Comments, Other Stuff}
\halign{#\hfil\qquad&#\hfil\cr
\cs{footnote\{{\rm text}\}}&numbered footnote\cr
{\tt\%}&comment out a line\cr
\cs{begin\{comment\}\dots\cs{end}\{\dots\}}
    \quad long comment {(\ss verbatim)}\hcr
\cs{typeout\{{\rm text}\}}&print to terminal\cr
\cs{typein\{{\rm text}\}}&get input from keyboard\cr
\cs{typein[\cs{cmd}]\{{\rm text}\}}&assign input to \cs{cmd}\cr
\cs{protect}&protects fragile commands\cr
\cs{-}&optional hyphen\cr
\multispan2%
\cs{hyphenation\{{\rm hypenated words}\}}\quad extra hyphenated words\cr
}

\copyrightnotice 

%%%%%%%%%%%%%%%% START OF PAGE 2 %%%%%%%%%%%%%%%%%%%

\section{Dimensions, Spacing, and Glue}
\begintext
Dimensions are specified as  \<number>\<unit of measure>.
\par
Glue is specified as \<dimen> {\tt plus}\<dimen> {\tt minus}\<dimen>.
\endtext
\par
\vbox{%
\offinterlineskip
\halign to\hsize{%
#\tabskip=1em&\tt#\hfil\tabskip=0pt plus1fil&\vrule#
&#\hfil\tabskip=1em&\tt#\hfil\tabskip=0pt plus1fil&\vrule#
&#\hfil\tabskip=1em&\tt#\hfil\tabskip=0pt plus1fil&\vrule#
&#\hfil\tabskip=1em&\tt#\hfil\tabskip=0pt\cr
point&pt&&pica&pc&&inch&in&&centimeter&cm\cr
\omit&& height 1.5pt&&& height 1.5pt&&& height 1.5pt\cr
m width&em&&x height&ex&&math unit&mu&&millimeter&mm\cr
\omit&& height 1.5pt&&& height 1.5pt&&& height 1.5pt\cr
\multispan2 1 pc = 12 pt\hfil&&
\multispan2 1 in = 72.72 pt\hfil&&
\multispan2 2.54 cm = 1 in\hfil&&
\multispan2 18 mu = 1 em\hfil\tabskip=0pt\cr
}}
\halign{#\hfil\qquad&#\hfil\cr
\cs{ }\ \ \cs{quad}\ \ \cs{qquad}&white space (1 space, 1 em, 2 em)\cr
\cs{hspace\{10pt\}}&specified horizontal space\cr
\cs{hspace*\{10pt\}}&space even at line start\cr
}
\halign{%
#\quad\hfil&#\hfil\cr
Horizontal Spacing (Math):&
   \cs{,} thin space  \quad
   \cs{:} med space
\cr
\multispan2\hfil
   \cs{;} thick space \quad
   \cs{!} neg.\ thin space \quad
   \cs{mspace}\<muglue>\cr
}
\halign{#\hfil\qquad&#\hfil\cr
\cs{strut},\cs{mathstrut}&invisible vertical space\cr
\cs{phantom\{\dots\}}&invisible space\cr
\cs{vphantom\{\dots\}}&invisible vertical space\cr
\cs{smash[bt]\{\dots\}}&typeset w/zero height,depth\cr
\cs{hfill}&fill with space\cr
\cs{dotfill}&fill with dots\cr
\cs{hrulefill}&fill with rule (line)\cr
\cs{par}&new paragraph\cr
\cs{newline} or \cs{\\}&force a new line\cr
\cs{\\*}&new line, prohibit page break\cr
\cs{\\[5pt]}&new line skipping 5 pts\cr
\cs{vspace\{1in\}}&specified vertical space\cr
\cs{vspace*\{1in\}}&space even at page start\cr
\cs{newpage}&force a new page\cr
}
\halign{#\hfil\quad&#\hfil\cr
$\bullet$ \bf  Length Variables\hcr
\multispan2%
\cs{newlength\{\cs{lngth}\}}\hfill create length varible \cs{lngth}\cr
\cs{setlength\{\cs{lngth}\}\{{\rm dimen}\}}&set value of \cs{lngth}\cr
\cs{addtolength\{\cs{lngth}\}\{{\rm dimen}\}}&increase \cs{lngth}\cr
}
\halign{#\hfil\quad&#\hfil\cr
$\bullet$ \bf  Useful Length Assignments\hcr
\cs{enlargethispage\{\cs{baselineskip}\}}
   &force extra line\cr
\cs{setlength\{\cs{hangindent}\}\{30pt\}}
   &indentation\cr
\cs{setlength\{\cs{hangafter}\}\{3\}}
   &indent after\cr
\cs{renewcommand\{\cs{baselinestretch}\}\{2\}}
   doublespaced\hcr
}

\section{Accents}
% ***** Four Column Format *****
\halign to\hsize{%
\tabskip=\centering
#\hfil & \hfil$#$\hfil & #\hfil & \hfil#\hfil\tabskip=0pt\cr
%----------- 4 Column Data -------------------
Type & \omit\hfil Example\hfil & In Math & In Text \cr
hat & \hat a & \cs{hat} & \cs{\char`\^} \cr
expanding hat & \widehat{abc} & \cs{widehat} & none\cr
check & \check a & \cs{check} & \cs{v}\cr
tilde & \tilde a & \cs{tilde} & \cs{\char`\~} \cr
expanding tilde & \widetilde{abc} & \cs{widetilde} & none\cr
acute & \acute a & \cs{acute} & \cs{\char`\'} \cr
grave & \grave a & \cs{grave} & \cs{\char`\`} \cr
dot & \dot a & \cs{dot} & \cs{.} \cr
double dot & \ddot a & \cs{ddot} & \cs{\char`\"} \cr
breve & \breve a & \cs{breve} & \cs{u} \cr
bar & \bar a & \cs{bar} & \cs{=} \cr
vector & \vec a & \cs{vec} & none\cr
cedilla & \hbox{\c c} & none & \cs{c}\cr
}

%%%%%%%% Definitions for "Additional Text Symbols" section %%%%%%%
\def\textvisiblespace{%
\vrule height 2pt width.4pt depth1pt%
\vrule height-.6pt width7pt depth1pt%
\vrule height 2pt width.4pt depth1pt}
\def\textbullet{$\bullet$}
\def\textcircled#1{{\ooalign{%
     \hfil\raise.07ex\hbox{#1}\hfil\crcr\mathhexbox20D}}}

\section{Additional Text Symbols}
\halign{#\hfil\quad&\hfil#\hfil\qquad
   &#\hfil\quad&\hfil#\hfil\qquad&#\hfil\quad&\hfil#\hfil\cr
\cs{dag}&\dag &\cs{copyright}&\copyright &\cs{pounds}&{\it\$}\cr
\cs{ddag}&\ddag &\cs{textcircled\{r\}}&\textcircled{r}\cr
\cs{P}&\P &\cs{textvisiblespace}&\textvisiblespace\cr
\cs{S}&\S &\cs{textbullet}&\textbullet\cr
%\cs{pounds}&{\it\$}&\cs{textperiodcentered}&$\cdot$\cr
}

\section{Fonts}
\halign{%
#\hfil\hglue.5em&#\hfil&\hglue.5em#\hfil\cr
$\bullet$ {\bf Text Fonts}\hcr
\cs{textnormal\{\dots\}}&{\tt\{}\cs{normalfont\dots\}}
    &document font\cr
\cs{textrm\{\dots\}}&{\tt\{}\cs{rmfamily\dots\}}
    &roman\cr
\cs{textsf\{\dots\}}&{\tt\{}\cs{sffamily\dots\}}
    &\ss sans serif font\cr
\cs{texttt\{\dots\}}&{\tt\{}\cs{ttfamily\dots\}}
    &\tt typewriter style\cr
\cs{textbf\{\dots\}}&{\tt\{}\cs{bfseries\dots\}}
    &\bf bold\cr
\cs{textup\{\dots\}}&{\tt\{}\cs{upshape\dots\}}
    &upright\cr
\cs{textit\{\dots\}}&{\tt\{}\cs{itshape\dots\}}
    &\it italic\cr
\cs{textsl\{\dots\}}&{\tt\{}\cs{slshape\dots\}}
    &\sl slanted\cr
\cs{textsc\{\dots\}}&{\tt\{}\cs{scshape\dots\}}
    &S{\smallfont MALL} C{\smallfont APITALS}\cr
\cs{emph\{\dots\}}&{\tt\{}\cs{em\dots\}}
    &\it emphasize\cr
\cs{fbox\{\dots\}}&&\boxit{framed text}\cr
$\bullet$ {\bf Font Environments} exist for above types,
   e.g.,\hcr
\cs{begin\{ttfamily\}\dots\cs{end}\{\dots\}}\hcr
$\bullet$ \bf Changing Font Sizes\hcr
\multispan3\hfil
\cs{tiny}, \cs{scriptsize}, \cs{footnotesize}, \cs{small}\hfil\cr
\multispan3\hfil
\cs{normalsize}
\cs{large}, \cs{Large}, \cs{LARGE}, \cs{huge}, \cs{Huge}\hfil\cr
}
\halign{#\hfil\quad&#\hfil\cr
$\bullet$ {\bf Math Fonts}\hcr
\cs{mathrm\{\dots\}}&roman\cr
\cs{mathbf\{\dots\}}&{\bf bold} (letters)\cr
\cs{boldsymbol\{\dots\}}&{\bf bold} (symbol) {\ss(amsmath)}\cr
\cs{mathit\{\dots\}}&\it italic\cr
\cs{mathcal\{\dots\}}&caligraphic $\cal A, B, C$\cr
\cs{usepacakge\{eucal\}}
   &redef \cs{mathcal} to script $\script{A}$, $\script{B}$, $\script{C}$\cr
\cs{mathfrak\{\dots\}}
     &Fraktur $\frak{A}$, $\frak{a}$, $\frak{B}$, $\frak{b}$  {\ss(amsfonts)}\cr
\cs{mathbb\{\dots\}}&Blackboard bold $\Bbb A$, $\Bbb B$, $\Bbb C$ 
{\ss(amsfonts)}\cr
\cs{boxed\{\dots\}}&\boxit{framed math}\cr
$\bullet$ \bf Math Font Sizes\hcr
\cs{displaystyle}&display size\cr
\cs{textstyle}&text size\cr
\cs{scriptsize}&sub/superscript size\cr
\cs{scriptscriptsize}&doubly sub/superscripted size\cr
}


\section{Boxes}
\halign{#\hfil\qquad&#\hfil\cr
\cs{mbox\{\dots\}}&one line of text\cr
\cs{text\{\dots\}}&one line of text {(\ss amsmath)}\cr
\cs{parbox\{{\rm width}\}\{{\rm text}\}}&paragraph of text\cr
\cs{parbox[{\rm align}][{\rm height}][{\rm inner align}]%
    \{{\rm width}\}\{{\rm text}\}}\hcr
\cs{marginpar\{\dots\}}&marginal comment\cr
\cs{rule[-1pt]\{20pt\}\{10pt\}}&solid box
\vrule height7.2pt depth.8pt width16pt\cr
\cs{raisebox\{5pt\}\{{\rm text}\}}&raised box\cr
}
\halign{#\hfil\quad&#\hfil\cr
\cs{makebox[{\rm width}][{\rm alignment}]\{{\rm text}\}}
      &box of text\cr
\cs{framebox[{\rm width}][{\rm alignment}]\{{\rm text}\}}
     &\boxit{framed text}\cr
}
\halign{#\hfil\quad&#\hfil\cr
\cs{setlength\{\cs{fboxsep}\}\{5pt\}}&space around text\hcr
\cs{setlength\{\cs{fboxrule}\}\{3pt\}}&width of box borders\cr
}

\section{Overfull and Underfull Boxes}
\halign{#\hfil\qquad&#\hfil\cr
{\ss draft}&document class marks overfulls\cr
\cs{overfullrule}&width of overfull marker\cr
\cs{begin\{setlength\}\{\cs{hfuzz}\}\{2pt\}\dots\cs{end}\{\dots\}}
\hcr
&\hfill allow small overfulls\cr
}

\section{Multicolumn Printing}
\halign{#\hfil\qquad&#\hfil\cr
\cs{twocolumn}&double column on new page\cr
\cs{onecolumn}&single column on new page\cr
}
\leftline{%
\cs{begin\{multicols\}\{\hskip-2pt
$n$\}[{\rm title}]\dots\cs{end}\{\dots\}}}
\rightline{%
multicolumn environment {(\ss multicol)}\qquad}

\section{Array and Tabular Environments}
\halign{#\hfil\quad&#\hfil\cr
\cs{begin\{tabular\}[{\rm POS}]%
   \{{\rm COLS}\}\dots\cs{end}\{\dots\}}
   \hcr
\cs{begin\{array\}[{\rm POS}]%
   \{{\rm COLS}\}\dots\cs{end}\{\dots\}}
   \hcr
Use {\tt tabular} for text, {\tt array} for mathematics
   \hcr
\tt{\&}, \cs{\\ }&column and row separators\cr
POS aligns top ({\tt t}), bottom ({\tt b}), center (default)
   \hcr
COLS gives formats for columns:
   \hcr
\qquad\tt{l},\tt{c},\tt{r}&left, center, right justified\cr
\qquad\tt{|}&vertical rule\cr
\qquad\tt{@\{\dots\}}&material between columns\cr
\qquad\tt{@\{\}}&no space between columns\cr
\qquad\tt{*\{n\}\{\dots\}}&$n$ copies of material\cr
\qquad\tt{p\{{\rm width}\}}&set column width\cr
\cs{hline}&horizontal line between rows\cr
\cs{cline\{i-j\}}&line across columns $i$ to $j$\cr
\cs{multicolumn\{n\}\{COLS\}\{\dots\}}  \hcr
&\hfill  span $n$ columns using format in COLS\cr
\cs{setlength\{\cs{tabcolsep}\}\{0pt\}}\quad set column separation\hcr
\cs{setlength\{\cs{itemsep}\}\{0pt\}}\quad set item separation\hcr
\cs{renewcommand\{\cs{arraystretch}\}\{1.25\}}
           \quad open up array\hcr
%%
%%
$\bullet$ \bf Example of a table using \cs{tabular}\hcr
\cs{begin\{table\}} \hcr
\quad \cs{begin\{center\}}  \hcr
\qquad \cs{begin\{tabular\}\{|l|c|c|\} \cs{hline}}
       \hcr
\qquad\quad \tt{Name \& Exam \& Grade \\\\ \cs{hline}}
       \hcr
\qquad\quad \tt{Dan \& 97\\\% \& A \\\\ \cs{hline}}
       \hcr
\qquad \cs{end\{tabular\}} \hcr
\qquad \cs{caption\{Math 101 Final Grades\}} \hcr
\qquad \cs{label\{GradeTable\}} \hcr
\quad \cs{end\{center\}}\hcr
\cs{end\{table\}} \hcr
}

\vskip-20pt
\line{\hfil\hfil\hfil%
\hbox{
$\vbox{\offinterlineskip
\halign{\strut#&\vrule#&\enspace#\enspace\hfil%
&\vrule#&\hfil\enspace#\enspace\hfil%
&\vrule#&\hfil\enspace#\enspace\hfil&\vrule#\cr%
\noalign{\hrule}
\omit&height 1pt&&&&&&\cr
&&Name&&Exam&&Grade&\cr
\noalign{\hrule}
\omit&height 1pt&&&&&&\cr
&&Dan&&97\%&&A&\cr
\noalign{\hrule\vskip1\jot}
\multispan7\hfil\bf Math 101 Final Grades\hfil\cr
}}$}
\hfil}

\section{Tabbing Environment}
\halign{#\hfil\quad&#\hfil\cr
\cs{begin\{tabbing\}\dots\cs{end}\{\dots\}}
   &tabbing environment\cr
\cs{=}&set tab\cr
\cs{\\}&end line\cr
\cs{>}&move to next tab\cr
\cs{kill}&do not print line\cr
}

\section{File Suffixes and Types}
\halign{#\hfil\qquad&#\hfil\cr
$\bullet$  {\bf \smallLaTeX Source Files}\hcr
{\tt .tex}&File containing a \smallLaTeX document\hcr
{\tt .sty}, {\tt .cls}\quad\smallLaTeX\ style and document class files\hcr
%% {\tt .sty}&\smallLaTeX\ style file\hcr
%% {\tt .cls}&\smallLaTeX\ document class file\hcr
{\tt .fd}&Font definition file\hcr
$\bullet$  {\bf Files Written by \smallLaTeX} \hcr
&(See also B{\sevenrm IB}\TeX and M{\sevenrm AKE}I{\sevenrm NDEX})\hcr
{\tt .aux}&cross-referencing and list information\cr
{\tt .dvi}&device independent typeset file\cr
%% {\tt .idx}&list of index entries (used by MakeIndex)\cr
%% {\tt .ind}&index file created by MakeIndex\cr
{\tt .glo}&list of glossary entries\cr
{\tt .lof}&list of figures (read by \cs{listoffigures})\cr
{\tt .lot}&list of tables (read by \cs{listoftables})\cr
{\tt .toc}&table of contents (read by \cs{tableofcontents})\cr
%% {\tt .bib}&B{\sevenrm IB}\TeX\ bibliographic database file\cr
%% {\tt .bst}&B{\sevenrm IB}\TeX\ bibliographic style file\cr
%% {\tt .bbl}&B{\sevenrm IB}\TeX\ document bibliography file\cr
%% $\bullet$  {\bf \smallLaTeX\ Log Files}\hcr
{\tt .log}&\smallLaTeX\ log file\cr
%% {\tt .ilg}&MakeIndex index log file\cr
%% {\tt .blg}&B{\sevenrm IB}\TeX\ log file\cr
\cs{nofiles}\quad supresses all except {\tt.log} and {\tt.dvi}\hcr
}

\shortcopyrightnotice

%%%%%%%%%%%%%%%% START OF PAGE 3 %%%%%%%%%%%%%%%%%%%
\newcolumn

\title{\AMSLaTeX~Reference Card \#2}
\smallskip
\centerline{See the \TeX\ Reference Card for additional commands.}
\centerline{The notation {(\ss package)} indicates a required package.}

\section{Math Environments}
\halign{#\hfil\quad&#\hfil\cr
\cs{(\dots\cs{)}} or {\tt\$\dots\$}&inline math\cr
\cs{[\dots\cs{]}} or {\tt\$\$\dots\$\$}&displayed math\cr
\cs{begin\{equation\}\cs{label}\{{\rm eqname}\}\dots\cs{end}\{\dots\}}
     \hcr
   \multispan2\hfill numbered and labeled equation\cr
\cs{ref\{{\rm eqname}\}}&refer to labeled eqn\cr
\cs{mbox\{\dots\}}&text in math\cr
%
$\bullet$ The following require {\ss amsmath}\hcr
\cs{text\{\dots\}}&text in math\cr
\cs{begin\{equation*\}\dots\cs{end}\{\dots\}}&unnumbered eqn\cr
\multispan2 \cs{tag\{{\rm eqtag}\}}\hfill use eqtag instead of number\cr
\cs{notag}&supress equation tag\cr
\cs{eqref\{{\rm eqname}\}}&ref with parens\cr
\cs{begin\{subequations\}\dots\cs{end}\{\dots\}}\hcr
\multispan2\hfill group equations for numbering\cr
\cs{numberwithin\{equation\}\{section\}}\hcr
   \multispan2\hfill number equations within sections\cr
}

\section{Theorems, Lemmas, Etc.}
\halign{#\hfil\quad&#\hfil\cr
$\bullet$ \bf Defining Theorem-Like Environments\hcr
\cs{newtheorem\{{\rm name}\}\{{\rm label}\}}
   &theorem environment\cr
\cs{newtheorem*\{{\rm name}\}\{{\rm label}\}}
   &unnumbered {\ss(amsthm)}\cr
\cs{newtheorem\{{\rm name}\}[{\rm other name}]\{{\rm label}\}}
   \hcr
   \multispan2\hfill
   numbered consecutively with other environment\cr
\cs{newtheorem\{{\rm name}\}\{{\rm label}\}[section]}
   \hcr
   \multispan2\hfill
   numbered by section (or {\tt chapter}, etc.)\cr
\cs{swapnumbers}&put numbers on left\cr
$\bullet$ {\bf Theorem-Like Environment Styles}
     {\ss(amsthm)}\hcr
\cs{theoremstyle\{plain\}}&most emphatic\cr
\cs{theoremstyle\{defintion\}}&medium emphasis\cr
\cs{theoremstyle\{remark\}}&least emphatic\cr
$\bullet$ \bf Invoking Theorem-Like Environments\hcr
\cs{begin\{{\rm name}\}\dots\cs{end}\{\dots\}}
   &invoke environment\cr
\cs{begin\{{\rm name}\}[{\rm label}]\dots}
   &invoke with new label\cr
\multispan2%
If proclamation starts with a list, put in \cs{hfill}\hfill\cr
\cs{begin\{proof\}\dots\cs{end}\{\dots\}}&proof environment\cr
\multispan2\cs{begin\{proof\}[{\rm label}]\dots\cs{end}\{\dots\}}
   \hfill proof with label\cr
\cs{qedsymbol}&end of proof marker\cr
\multispan2%
\cs{renewcommand\{\cs{qedsymbol}\}\{\dots\}}\hfill redefine marker\cr
}

\section{Commutative Diagrams {\ss(amscd)}}
\halign{#\hfil\qquad&#\hfil\cr
Separate lines with \cs{\\}, do not use {\tt\&}s\hcr
\multispan2%
\cs{begin\{CD\}\dots\cs{end}\{CD\}}\hfill\qquad commutative diagram\cr
\tt@>\#1>\#2>&right arrow with labels\cr
\tt@<\#1<\#2<&left arrow with labels\cr
\tt@V\#1V\#2V&down arrow with labels\cr
\tt@A\#1A\#2A&up arrow with labels\cr
\tt@=&long horizontal equal sign\cr
\tt@|&long vertical equal sign\cr
\tt@.&leave out an arrow\cr
}

\section{Multiline Math Displays {\ss(amsmath)}}
\halign{\tt#\hfil\quad&#\hfil\cr
Use as \cs{begin\{{\rm command}\}\dots\cs{end}\{{\rm command}\}}
   \hcr
Separate items with {\tt\&}, separate lines with \cs{\\}
   \hcr
No \cs{\\} on last line, \cs{\\[{\rm dim}]} to skip space\hcr
\noalign{\vskip1\jot}
$\bullet$ \bf Full Math Environments (full line)\hcr
gather&centered, numbered equations\cr
gather*&centered, unnumbered equations\cr
multline&first line left, last line right, rest centered\cr
multline*&same as multline, but unnumbered\cr
align&formulas aligned at {\tt\&} signs\cr
align*&same as align, but unnumbered\cr
flalign&flush left and right align\cr
alignat&{\tt align} without space, needs\cr
&\hfill argument \cs{begin\{alignat\}\{{\rm\# of cols}\}}\cr
}
\halign{#\hfil\quad&#\hfil\cr
\cs{intertext\{{\rm text}\}}&text between lines\cr
\cs{shoveleft},\cs{shoveright}&move {\tt multline} line left, right\cr
\cs{allowdisplaybreaks}&allow page breaks (\cs{\\*} prohibits)\cr
\cs{displaybreak}&force page break (before \cs{\\})\cr
}
\halign{\tt#\hfil\quad&#\hfil\cr
$\bullet$ \bf Math Subenvironments (within math display)\hcr
gathered&centered equations\cr
aligned&formulas aligned at {\tt\&} signs\cr
split&split long formula within other environment\cr
cases&cases, with $\{$ on left\cr
matrix&matrix (of up to 10 columns)\cr
pmatrix, bmatrix, vmatrix, Vmatrix\hcr
\multispan2\hfill
matrix variants enclosed by $(\cdots)$, $[\cdots]$, $|\cdots|$, $\|\cdots\|$\cr
\cs{setcounter\{MaxMatrixCols\}\{12\}}\hcr
&\hfill   increase number of matrix columns\cr
\multispan2\cs{hdotsfor\{{\rm num}\}}\hfill dots across columns\cr
}

\section{Overlines, Underlines, and Arrows}
\halign{#\hfil\quad&#\hfil\cr
\cs{underline\{\dots\}}&underline\cr
\cs{overline\{\dots\}}&overline\cr
\cs{overbrace\{\dots\}\char`^\{\dots\}}&overbrace\cr
\cs{underbrace\{\dots\}\char`_\{\dots\}}&underbrace\cr
\cs{overightarrow\{\dots\}}&over right arrow\cr
\cs{overleftarrow\{\dots\}}&over left arrow\cr
\cs{overleftrightarrow\{\dots\}}&over left-right arrow\cr
\cs{underrightarrow\{\dots\}},\cs{underleftarrow\{\dots\}}, etc.\hcr
\cs{xrightarrow[{\rm bot}]\{{\rm top}\}}&stretchable w/sub/supscripts\cr
\cs{xleftarrow[{\rm bot}]\{{\rm top}\}}&stretchable w/sub/supscripts\cr
}


\section{Operator Names}
\halign to\hsize{#\hfil\tabskip=\centering
&#\hfil&#\hfil&#\hfil&#\hfil&#\hfil&#\hfil&#\hfil\tabskip=0pt\cr
\cs{arccos}&\cs{cos}&\cs{csc}&\cs{exp}&\cs{ker}&\cs{liminf}&\cs{min}&\cs{sinh}\cr
\cs{arcsin}&\cs{cosh}&\cs{deg}&\cs{gcd}&\cs{lg}&\cs{limsup}&\cs{Pr}&\cs{sup}\cr
\cs{arctan}&\cs{cot}&\cs{det}&\cs{hom}&\cs{lim}&\cs{log}&\cs{sec}&\cs{tan}\cr
\cs{arg}&\cs{coth}&\cs{dim}&\cs{inf}&\cs{ln}&\cs{max}&\cs{sin}&\cs{tanh}\cr
}
\halign{#\hfil\quad&#\hfil\cr
{\tt a \cs{equiv} b \cs{pmod\{m\}}}&$a\equiv b\quad(\hbox{mod}~{m})$\cr
{\tt a \cs{equiv} b \cs{mod\{m\}}}&$a\equiv b\quad\hbox{mod}~{m}$\cr
{\tt a \cs{bmod} m}&$a~\hbox{mod}~m$\cr
}
\halign{#\hfil\quad&#\hfil\cr
\cs{DeclareMathOperator\{\cs{\rm cmd}\}\{{\rm opname}\}}
   &create operator\cr
\cs{DeclareMathOperator*\{\cs{\rm cmd}\}\{{\rm opname}\}}
   &\qquad with limits\cr
\multispan2%
\cs{operatorname\{\dots\}}\hfill typeset as an operator\cr
\multispan2%
\cs{operatorname*\{\dots\}}\hfill with limits\cr
}

\section{Large Operators}
% ***** Three Column Format *****
\paralign to\hsize{%
$#$\hfil\SPC\tabskip=0pt&#\hfil\tabskip=0pt plus 1 fil
&$#$\hfil\SPC\tabskip=0pt&#\hfil\tabskip=0pt plus 1 fil
&$#$\hfil\SPC\tabskip=0pt&#\hfil\cr
%----------- 3 Column Data -------------------
\sum &\cs{sum}&\bigcap &\cs{bigcap}
      &\bigodot &\cs{bigodot}\cr
\prod &\cs{prod}&\bigcup &\cs{bigcup}
      &\bigotimes &\cs{bigotimes}\cr
\coprod &\cs{coprod}&\bigsqcup
      &\cs{bigsqcup}&\bigoplus &\cs{bigoplus}\cr
\int &\cs{int}&\bigvee &\cs{bigvee}
      &\biguplus &\cs{biguplus}\cr
\oint &\cs{oint}&\bigwedge &\cs{bigwedge}\cr
}
\halign{#\hfil\qquad&#\hfil\cr
\cs{substack\{{\rm xxx}\\\\ {\rm yyy}\}}
   &stacked sub or superscripts\cr
\cs{limits},\cs{nolimits}&force or forbid displayed limits\cr
\cs{oint},\cs{iint},\cs{iiint},\cs{iiiint},\cs{idotsint}
    \hcr
\multispan2\hfill integral variants {\ss(amsmath)}\cr
}


\section{Delimiters}
% ***** Three Column Format *****
\halign to\hsize{%
$#$\hfil\SPC\tabskip=0pt&#\hfil\tabskip=0pt plus 1 fil
&$#$\hfil\SPC\tabskip=0pt&#\hfil\tabskip=0pt plus 1 fil
&$#$\hfil\SPC\tabskip=0pt&#\hfil\cr
%----------- 3 Column Data -------------------
[&\cs{lbrack} or \cs{[}&\{&\cs{lbrace} or \cs{\{}&\langle&\cs{langle}\cr
]&\cs{rbrack} or \cs{]}&\}&\cs{rbrace} or \cs{\}}&\rangle&\cs{rangle}\cr
\vert&\cs{vert} or \cs{|}&\lfloor&\cs{lfloor}&\lceil&\cs{lceil}\cr
\|&\cs{Vert} or \cs{\char`\|}&\rfloor&\cs{rfloor}&\rceil&\cs{rceil}\cr
\uparrow&\cs{uparrow}&\Uparrow&\cs{Uparrow}&\updownarrow&\cs{updownarrow}\cr
\downarrow&\cs{downarrow}&\Downarrow&\cs{Downarrow}&\Updownarrow&\cs{Updownarrow}\cr
%%%[\mskip-.75\thinmuskip[
%%%  &\tt[\cs![&(\!(&\tt(\cs!(&\langle\!\langle&\cs{langle}\cs!\cs{langle}\cr
%%%]\mskip-.75\thinmuskip]
%%%  &\tt]\cs!]&)\!)&\tt)\cs!)&\rangle\!\rangle&\cs{rangle}\cs!\cs{rangle}\cr
}
\halign{#\hfil\qquad&#\hfil\cr
\cs{left(}\quad\cs{right)}&expanding delimiters\cr
\cs{left.}\quad\cs{right.}&empty delimiters\cr
\cs{bigl(}\quad\cs{bigr)}&big delimiters\cr
\cs{Bigl(}\quad\cs{Bigr)}&bigger delimiters\cr
\cs{biggl(}\quad\cs{biggr)}&even bigger delimiters\cr
\cs{bigm|},\cs{biggm|}&big binary relation delimiters\cr
}


\section{Roots}
\halign{#\hfil\quad&#\hfil\cr
\cs{sqrt\{\dots\}}&square root $\sqrt{\phantom{x}}$\cr
\cs{sqrt[$n$]\{\dots\}}&$n$th root
     \smash{$\root n\of{\phantom{x}}$}\cr
\cs{leftroot\{2\},\cs{uproot}\{2\}}&move root left or up\cr
}

\section{Ellipses}
\halign{#\hfil\quad&#\hfil\cr
\cs{ldots},\cs{cdots},\cs{dots}&ellipses\cr
\cs{vdots},\cs{ddots}&vertical and diagonal dots\cr
\multispan2\cs{dotsc},\cs{dotsb},\cs{dotsm},\cs{dotsi}\quad\hfill
   more ellipses {\ss(amsmath)}\cr
}

\section{Fractions and Stacked Relations}
\halign{#\hfil\quad&#\hfil\cr
\cs{frac\{$n$\}\{$d$\}}&fraction $n\over d$\cr
\cs{dfrac\{$n$\}\{$d$\}}&displaystyle fraction\cr
\cs{tfrac\{$n$\}\{$d$\}}&textstyle fraction\cr
\cs{binom\{$n$\}\{$d$\}}&binomial coefficient $n\choose d$\cr
\cs{genfrac\{{\rm ldelim}\}\{{\rm rdelim}\}\{{\rm
thick}\}\{{\rm style}\}\{{\rm num}\}\{{\rm den}\}}\hcr
\cs{cfrac\{\dots\}\{\dots\}}&continued fraction\cr
\cs{stackrel\{{\rm top}\}\{{\rm bot}\}}&stacked relation\cr
\cs{overset\{{\rm top}\}\{{\rm bot}\}}&stacked symbol {\ss(amsmath)}\cr
\cs{underset\{{\rm bot}\}\{{\rm top}\}}&stacked relation {\ss(amsmath)}\cr
\cs{sideset\{\char`_\{{\rm ll}\}\char`^\{{\rm ul}\}\}%
\{\char`_\{{\rm lr}\}\char`^\{{\rm ur}\}\}\{{\rm largeop}\}}\hcr
\multispan2\hfill large operator with left/right sub/supscripts\cr
}

\section{Negated Relations}
\halign{#\hfil\quad&#\hfil\cr
\cs{not}&negate a relation\cr
\cs{ne}&not equal $\ne$\cr
\cs{notin}&not a member of $\notin$\cr
\cs{nmid}&not divisible \nmid\cr
}

\copyrightnotice

%%%%%%%%%%%%%%%% START OF PAGE 4 %%%%%%%%%%%%%%%%%%%

\section{User Defined Commands}
\halign{#\hfil\quad&#\hfil\cr
\multispan2%
\cs{newcommand\{\cs{name}\}\{{\rm replacement text}\}} \hfill new command\cr
\cs{newcommand\{\cs{name}\}[$n$]\{{\rm text with {\tt\#1,\#2,\dots,\#$n$}}\}}
   \hcr
\multispan2\hfill new command with $n$ arguments \cr
Example: \cs{newcommand\{\cs{vect}\}[2]\{\#1\_1,\cs{ldots},\#1\_\{\#2\}\}}
     \hcr
\cs{newcommand\{\cs{name}\}[$n$][{\rm default}]\{\dots\}}\hcr
   \multispan2\hfill command with args and default value for {\tt\#1}\cr
\cs{renewcommand\{\dots\}\{\dots\}}&redefine existing command\cr
\cs{providecommand\{\dots\}\{\dots\}}&define if doesn't exist\cr
\cs{newcommand*\{\dots\}\{\dots\}}&command with one par arg\cr
\cs{ensuremath\{\dots\}}&forces math mode\cr
\cs{show\cs{command}}&print definition of \cs{command}\cr
\cs{showthe\cs{paramname}}&print value of a parameter\cr
}

\section{User Defined Environments}
\halign{#\hfil\quad&#\hfil\cr
\cs{newenvironment\{{\rm name}\}\{{\rm pretext}\}\{{\rm posttext}\}}
      &\hbox to4em{\hfill}\cr
\multispan2\hfill new environment with material before and after\cr
\cs{newenvironment[$n$]\{{\rm name}\}\{\dots\}\{\dots\}}
    \hcr
\multispan2\hfill environment with $n$ arguments\cr
\cs{newenvironment[$n$][{\rm default}]\{{\rm name}\}\{\dots\}\{\dots\}}
    \hcr
\multispan2\hfill environment with default value for {\tt\#1}\cr
\multispan2%
\cs{renewenvironment\{{\rm name}\}\{\dots\}\{\dots\}}\hfill
redefine envrment\cr }

\section{M{\tenbf AKE}I{\tenbf NDEX}}
\halign{#\hfil\qquad&#\hfil\cr
$\bullet$ {\bf MakeIndex} File  Suffixes\hcr
{\tt .idx}, {\tt .ind}, {\tt .ilg}
&entry listing, index file, log file\cr
}
%% \halign{#\hfil\qquad&#\hfil\cr
%% $\bullet$ {\bf MakeIndex} File  Suffixes\hcr
%% {\tt .idx}&MakeIndex entry listing file\cr
%% {\tt .ind}&MakeIndex index file\cr
%% {\tt .ilg}&MakeIndex index log file\cr
%% }
\halign{#\hfil\quad&#\hfil\cr
$\bullet$ {\bf MakeIndex} Commands in Document File\hcr
\cs{usepackage\{makeidx\}}\qquad use indexing package\hcr
\multispan2\hfill (Do not include this line if using AMS packages.)\cr
\cs{makeindex}&tell \smallLaTeX\ to create an {\tt .idx} file\cr
\cs{printindex}&tell \smallLaTeX\ to print index here\cr
\cs{nofiles}&supresses creation of {\tt .idx} and {\tt .glo} files\cr
}
\halign{#\hfil\qquad&#\hfil\cr
$\bullet$ Creating {\bf MakeIndex} {\tt.idx} File\hcr
\cs{index\{{\rm entry}\}}&main entry\cr
\cs{index\{{\rm entry}!{\rm entry}\}}&subentry\cr
\cs{index\{{\rm entry}!{\rm entry}!{\rm entry}\}}& subsubentry\cr
\cs{index\{{\rm text}@{\rm entry}\}}&with placement info\cr
\cs{index\{{\rm entry}|see\{{\rm entry}\}\}}&cross referenced entry\cr
\multispan2%
\cs{index\{{\rm entry}|{\rm modifier}\}}\hfill entry with page modifier\cr
\multispan2\hfill
e.g.~\cs{index\{gnats|textbf\}} give bold page number\cr
\multispan2%
\cs{index\{{\rm entry}|(\}} \dots \cs{index\{{\rm entry}|)\}}
   \hfill page range\cr
Special Characters: \qquad{\tt\char`\"!}\quad {\tt\char`\"@}\quad
     {\tt\char`\"|}\quad {\tt\char`\"\char`\"}\hcr
}
\halign{#\hfil\cr
$\bullet$   Creating An Index With {\bf MakeIndex}\cr
(1)\quad Typeset document containing \cs{makeindex} command.\cr
(2)\quad Run MakeIndex on {\tt.idx} file to create {\tt.ind} file.\cr
(3)\quad Typeset document containing \cs{printindex} command.\cr
}

\section{Glossary}
\halign{#\hfil\quad&#\hfil\cr
\cs{makeglossary}&tell \smallLaTeX\ to create a {\tt .glo} file\cr
\cs{glossary\{{\rm entry}\}}&create a glossary entry\cr
\cs{glossaryentry\{{\rm entry}\}\{{\rm page no.}\}}
  \quad entries in {\tt .glo} file\hcr
\cs{input filename.glo}&read glossary file\cr
User must define \cs{makeglossary}, e.g.,\hcr
\qquad \cs{newcommand\{\cs{glossaryentry}\}[2]\{\#1, page \#2\\par\}}\hcr
}

\section{Time and Date}
\halign{#\hfil\quad&#\hfil\cr
\cs{today}&current date\cr
Use \cs{the} to display the following items\hcr
\cs{day}, \cs{month}, \cs{year},
\cs{time}  (minutes since midnight)\hcr
}

\section{Counters}
\halign{#\hfil\quad&#\hfil\cr
\multispan2%
\cs{newcounter\{cntr\}}\hfill create new counter named {\tt cntr}\cr
\multispan2%
\cs{newcounter\{cntr\}[cntr1]}\hfill   reset {\tt cntr} when {\tt cntr1}
      changes\cr
\cs{setcounter\{cntr\}\{{\rm value}\}}&set value of {\tt cntr}\cr
\cs{stepcounter\{cntr\}}&increment {\tt cntr}\cr
\cs{refstepcounter\{cntr\}}&increment and reset \cs{label}\cr
\cs{addtocounter\{cntr\}\{$n$\}}&increment by $n$\cr
\cs{value\{cntr\}}&value stored in \cs{cntr}\cr
\cs{thecntr}&the  value of {\tt cntr}\cr
\multispan2%
{\ss calc}\hfill package to do counter arithmetic\cr
$\bullet$ {\bf Counter Styles}\hcr
\multispan2\hfill
\cs{arabic\{\}} \cs{roman\{\}} \cs{Roman\{\}}
     \cs{alph\{\}} \cs{Alph\{\}} \hfill\cr
$\bullet$ {\bf Standard Counters}\hcr
\multispan2\hfill
\tt equation footnote figure page table \hfill\cr
\multispan2\hfill
\tt part chapter section subsection subsubsection \hfill\cr
\multispan2\hfill
\tt paragraph subparagraph enumi enumii enumiii enumiv  \hfill\cr
}
\halign{#\hfil\qquad&#\hfil\cr
\tt secnumdepth&depth to which sections are numbered\cr
\tt tocdepth&depth to which sections are put into toc\cr
}

\section{Customized List Environments}
\halign{&#\hfil\qquad\cr
\cs{begin\{list\}\{{\rm default label}\}\{{\rm declarations}\}}\hcr
&\cs{item} item 1 text\hcr
&\cs{item} item 2 text\hcr
\cs{end\{list\}}\hcr
\cs{begin\{trivlist\}\dots\cs{end}\{trivlist\}}\hcr
&&list with no labels or declarations, trivial lengths\hcr
}
\halign{#\hfil\quad&#\hfil\cr
$\bullet${\bf Declarations}\hcr
\cs{setlength\{{\rm length parameter}\}\{{\rm length}\}}\hcr
\cs{usecounter\{{\rm counter name}\}}\hcr
[Create counter first using \cs{newcounter\{{\rm counter name}\}}.]\hcr
}
\halign{#\hfil\quad&#\hfil\cr
$\bullet${\bf Length Parameters}
\enspace (see page 113 of Lamport for more)
\hcr
\cs{topsep}&separate preceding text and first item\cr
%% \cs{parsep}&separate paragraphs within items\cr
\cs{itemsep}&separate items\cr
\cs{leftmargin}&indent of item box from left margin\cr
%%\cs{rightmargin}&indent of item box from right margin\cr
\cs{labelwidth}&width of box for item label\cr
%% \cs{itemindent}&indent of label box from left margin\cr
\cs{labelsep}&separate label box from item box\cr
%% \cs{listparindent}&indent item paragraphs\cr
}

\section{The {\headingfonttt picture} Environment}
\halign{&#\hfil\quad\cr
\cs{begin\{picture\}($w$,$h$)\dots}\cs{end\{picture\}}&picture\cr
\cs{begin\{picture\}($w$,$h$)($\Delta x$,$\Delta y$)\dots}
        &\quad with offset\cr
\cs{put($x$,$y$)\{{\rm picture object}\}}&place object\cr
\cs{multiput($x$,$y$)($\Delta x$,$\Delta y$)\{$n$\}\{{\rm object}\}}
        &\quad $n$ times\cr
}
\halign{&#\hfil\quad\cr
{\bf Picture Objects:}\hcr
\quad \cs{makebox($x$,$y$)[tblr]\{{\rm text}\}}&box with text\cr
\quad \cs{line($\Delta x$,$\Delta y$)\{{\rm $x$ length}\}}
        &line of slope $\Delta y/\Delta x$\cr
\quad \cs{vector($\Delta x$,$\Delta y$)\{{\rm $x$ length}\}}
        &arrow of slope $\Delta y/\Delta x$\cr
\quad \cs{circle\{$r$\}}&circle of radius $r$\cr
\quad \cs{circle*\{$r$\}}&filled circle\cr
\quad \cs{oval($x$,$y$)[lrtb]}&oval (part or whole)\cr
\quad \cs{shortstack\{{\rm abc}\\\\{\rm xyz}\\\\\}}&stacked text\cr
\quad \cs{framebox($x$,$y$)[tblr]\{{\rm text}\}}&framed text\cr
\quad \cs{frame\{{\rm text}\},fbox\{{\rm text}\}}&other framed boxes\cr
\quad \cs{dashbox\{$d$\}($x$,$y$)\{{\rm text}\}}&dashed box\cr
\quad \cs{qbezier($x_1$,$y_1$)($x_2$,$y_2$)($x_3$,$y_3$)}
         &quadratic curve\cr
\quad \cs{savebox\{\cs{name}\}($x$,$y$)\{\dots\}}&store material\cr
\quad \cs{usebox\{\cs{name}\}}&retrieve material\cr
}
\halign{&#\hfil\quad\cr
\cs{graphpaper[$n$]\{$x$,$y$\}\{$w$,$h$\}}&print grid (\ss{graphpap})\cr
\cs{setlength\{\cs{unitlength}\}\{1pt\}}&change size of picture\cr
\cs{thinlines,\cs{thicklines}}&adjust line thickness\cr
}



\section{Color \ss{(color)}}
\halign{&#\hfil\quad\cr
\cs{color\{{\rm color}\}}&change color\cr
\cs{textcolor\{{\rm color}\}\{{\rm text}\}}&colored text\cr
\cs{colorbox\{{\rm color}\}\{{\rm text}\}}&colored background\cr
\cs{fcolorbox\{{\rm col$_1$}\}\{{\rm col$_2$}\}\{{\rm text}\}}
       &colored border \& background\cr
\cs{setlength\{\cs{fboxsep}\}\{5pt\}}&put space around text\cr
\cs{setlength\{\cs{fboxrule}\}\{3pt\}}&width of border of box\cr
\cs{pagecolor\{{\rm color}\}}&set background color of page\cr
\cs{definecolor\{{\rm name}\}\{rgb\}\{$r,g,b$\}}&
    define an RGB color\cr %%  $0\le r,g,b\le 1$\cr
\cs{definecolor\{{\rm name}\}\{cmyk\}\{$c,m,y,k$\}}\quad
    define a CMYK color\hcr %% $0\le c,m,y,k\le 1$\cr
{\bf Predefined Colors}\hcr
\quad{\tt black, white, red, green, blue, yellow, cyan, magenta}\hcr
}
%% \setlength{\fboxsep}{20pt}
%% \fcolorbox{Green}{Red}{\parbox{.5\hsize}{\color{Blue}
%% How are you? How are you? How are you? How are you? How are you? Fine!
%% \color{Black}}}
%% \setlength{\fboxrule}{.3mm}%
\vfill\vfill\vfill
\section{B{\tenbf IB}\hglue-1pt\TeX}
\halign{#\hfil\qquad&#\hfil\cr
$\bullet$ {\bf B{\sevenbf IB}\TeX} File  Suffixes\hcr
{\tt .bib}&B{\sevenrm IB}\TeX\ bibliographic database file\cr
{\tt .bst}&B{\sevenrm IB}\TeX\ bibliographic style file\cr
{\tt .blg}&B{\sevenrm IB}\TeX\ log file\cr
{\tt .bbl}&B{\sevenrm IB}\TeX\ document bibliography file\cr
}
\halign{#\hfil\quad&#\hfil\cr
$\bullet$ {\bf B{\sevenbf IB}\TeX} Commands in Document File\hcr
\cs{bibliographystyle\{{\rm bib style file}\}}\hcr
\quad Examples:\ 
  {\tt plain}, {\tt amsplain}, {\tt unsrt}, {\tt alpha}, {\tt abbrv}\hcr
\cs{bibliography\{{\rm bib database file(s)}\}} \hcr
\cs{cite\{{\rm label}\}}&cite a reference\cr  
\cs{nocite\{{\rm label}\}}&include ref in bib without citation\cr
\cs{nocite\{*\}}&include all references in bibliography\cr
%% {\tt mrabbrev.bib}&AMS file with math journal abbreviations\cr
}
\halign{#\hfil\cr
$\bullet$ Creating {\bf B{\sevenbf IB}\TeX} Database File\hcr
{\tt @STRING\{name = \char`\"{\rm text}\char`\"\}}
   \qquad define an abbreviation\cr
Put braces around non-initial capitalized title words.\cr
Use {\tt and} to separate multiple authors in {\tt author} field\cr
}
\halign{#\hfil\quad&#\hfil\cr
\qquad $\bullet${\bf General Format of a Database Entry}\hcr
\tt @ENTRYTYPE\{{\rm label},\cr
\qquad\tt fieldtype1 = \{{\rm entry1}\},\cr
\qquad\tt fieldtype2 = \{{\rm entry2}\},\cr
\noalign{\vskip-1\jot}
\qquad\qquad $\vdots$\cr
\noalign{\vskip-1\jot}
\quad\tt\}\cr
}
\halign{&#\hfil\quad\cr
\qquad $\bullet${\bf Database Entry Types}\hcr
\tt @ARTICLE\{\dots\} & \tt @MASTERSTHESIS\{\dots\}\cr
\tt @BOOK\{\dots\} & \tt @MISC\{\dots\}\cr
\tt @BOOKLET\{\dots\} & \tt @PHDTHESIS\{\dots\}\cr
\tt @INBOOK\{\dots\} & \tt @PROCEEDINGS\{\dots\}\cr
\tt @INCOLLECTION\{\dots\} & \tt @TECHREPORT\{\dots\}\cr
\tt @INPROCEEDINGS\{\dots\} & \tt @UNPUBLISHED\{\dots\}\cr
\tt @MANUAL\{\dots\} & \tt @COMMENT\{\dots\} \cr
}
\halign{&#\hfil\quad\cr
\qquad $\bullet${\bf Field Types Within Entries}\hcr
\vbox{\halign{#\hfil\cr
   \tt address \cr
   \tt author \cr
   \tt booktitle \cr
   \tt chapter \cr
   \tt crossref \cr
   \tt edition \cr
}}
\vbox{\halign{#\hfil\cr
   \tt editor \cr
   \tt howpublished \cr
   \tt institution \cr
   \tt journal \cr
   \tt key \cr
   \tt language \cr
}}
\vbox{\halign{#\hfil\cr
   \tt month \cr
   \tt note \cr
   \tt number \cr
   \tt organization \cr
   \tt pages \cr
   \tt publisher \cr
}}
\vbox{\halign{#\hfil\cr
   \tt school \cr
   \tt series \cr
   \tt title \cr
   \tt type \cr
   \tt volume \cr
   \tt year \cr
}}
\cr
}
\halign{#\hfil\cr
$\bullet$   Creating Document Bibliography With {\bf B{\sevenbf IB}\TeX}\cr
(1)\quad Typeset document to get new {\tt.aux} file.\cr
(2)\quad Run B{\sevenrm IB}\TeX\ on {\tt.aux} file to create {\tt.bbl} file.\cr
(3)\quad Retypeset document twice.\cr
}


\shortcopyrightnotice

\bye


%%%%%%%%%%%% ADDITIONAL SECTIONS %%%%%%%%%%%%%%%%%%%%

%%%%%%% POSSIBLE ADDITIONS %%%%%%%%%%
% AMS bibliography package
%%%%%%%%%%%%%%%%%%%%%%%%%%%%%%%%%%%%%

%% DELETED SECTIONS

\section{The {\headingfonttt slide} Document Class}
\halign{&#\hfil\qquad\cr
\cs{documentclass\{slides\}}\hcr
\cs{begin\{slide\}}\hcr
&\cs{begin\{center\}}\hcr
&&\cs{emph\{{\rm Slide Title}\}}\hcr
&\cs{end\{center\}}\hcr
&Slide material\hcr
\cs{end\{slide\}}\hcr
}
\halign{&#\hfil\quad\cr
\cs{begin\{overlay\}\dots\cs{end}\{\dots\}}& overlay slide\hcr
\cs{begin\{note\}\dots\cs{end}\{\dots\}}& one page note\hcr
}
\halign{&#\hfil\quad\cr
\cs{onlyslides\{4,8-12,19\}}&print only specified slides\hcr
\cs{onlynotes\{2,8-99\}}&print only specified notes\hcr
}

\section{The {\headingfonttt letter} Document Class}
\halign{#\hfil\qquad&#\hfil\cr
\cs{documentclass\{letter\}}\cr
\cs{address\{\dots\\\\\dots\\\\\dots\}}\cr
\cs{signature\{\dots\}}\cr
\cs{begin\{letter\}\{{\rm inner address}\}}\cr
\cs{opening\{{\rm Dear Mr. X}\}}\cr
\cs{closing\{{\rm Yours truly,}\}}\cr
\cs{cc\{\dots\\\\\dots\}}\cr
\cs{encl\{\dots\}}\cr
\cs{ps\{\dots\}}\cr
\cs{end\{letter\}}\cr
}


