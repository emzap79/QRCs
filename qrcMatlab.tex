%%%%%%%%%%%%%%
%  preamble  %
%%%%%%%%%%%%%%
% TeX Format

% version
\def\content{\uppercase{MatLab}}
\def\shortcontent{\content}
\def\versionnumber{0.1}  % Version of this reference card
\def\version{v\versionnumber\ \month\ \year}

% include stylefile
\input mydefs.sty
\mytitle

% shortcopyrightnotice
\shortintro
% \shortintroML

\section{Basic Commands and Functions}{}
\cmdS{Clc}	{clears command window}
\cmdS{clear}	{clears workspace}
\cmdS{Diary}	{creates a copy of all commands and most results}
\cmdS{exit}	{terminates MATLAB}
\cmdS{help}	{invokes help utility}
\cmdS{helpwin}	{invokes windowed help utility}
\cmdS{length}	{length of a vector or maximum dimension of an array}
\cmdS{size}	{display dimensions of a particular array}
\cmdS{quite}	{terminates MATLAB}
\cmdS{save}	{saves variables in a file}
\cmdS{who}	{lists  variables in memory}
\cmdS{whos}	{lists  variable names, sizes, and types in memory}

\section{Special Variables/Constants}{}
\cmdS{ans}	{default variable for last calculation}
\cmdS{eps}	{1}
\cmdS{flops}	{count of ßoating point operations}
\cmdS{Inf}	{infinity}
\cmdS{NaNa}	{NaN}
\cmdS{pi}	{the math pi (3.1415e)}
% \cmdS{i,j}	{\begin{math}\sqrt[]{-1}\end{math}}
\cmdS{realmax}	{largest real number MATLAB can represent}
\cmdS{realmin}	{smallest real number MATLAB can represent}
\cmdS{intmax}	{returns largest possible integer used in MATLAB}
\cmdS{intmin}	{returns smallest possible integer used in MATLAB}
\cmdS{clock}	{returns the time}
\cmdS{date}	{returns the date}

\section{Special Characters}{}
\cmdOper{[]}	{forms matrices}
\cmdOper{()}	{used in statements to group operations}
\cmdOper{.}	{decimal point}
\cmdOper{,! }	{separates subscripts or matrix elements}
\cmdOper{;}	{separates rows in a matrix definition or suppresses output}
\cmdOper{:}	{indicates all rows or all columns}
\cmdOper{=}	{assignment operator (not equality)}
\cmdOper{\%}	{indicates a comment}
\cmdOper{\%\%}	{cell divider}
\cmdOper{+}	{addition}
\cmdOper{-}	{ubtraction}
\cmdOper{*}	{multiplication}
\cmdOper{.*}	{array multiplication}
\cmdOper{/}	{division}
\cmdOper{./}	{array division}
\cmdOper{\expon}	{exponential}
\cmdOper{.\expon}	{array exponential}

\section{Relational and Logical Operators}{}
\cmdOper{<}	{less than}
\cmdOper{<==}	{less than or equal to}
\cmdOper{>}	{greater than}
\cmdOper{>==}	{greater than or equal to}
\cmdOper{==}	{equal to}
\cmdOper{~=}	{not equal to}
\cmdOper{\&}	{and}
\cmdOper{!}	{or}
\cmdOper{\tild}	{not}

\section{Conditional Statements}{}

% if, elseif, else
if expression\\
\hskip 7em statements\\
elseif expression\\
\hskip 7em statements\\
else expression\\
\hskip 7em statements\\
end


% \begin{verbatim}
% end
% switch switch_expression
%    case case_expression
%       statements
%    case case_expression
%       statements
%    otherwise
%       statements
% end
% \end{verbatim}

% \section{Loops}
% \begin{verbatim}
% for k = vectorOrColumnList
%    statements
% end

% while logicalExpression
%    statements
% end
% \end{verbatim}

%% Footer
\copyrightnotice

% Ending
\supereject
\if L\lr \else\null\vfill\eject\fi
\if L\lr \else\null\vfill\eject\fi
\bye

% EOF
