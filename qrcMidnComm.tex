%%%%%%%%%%%%%%
%  preamble  %
%%%%%%%%%%%%%%
% TeX Format

% version
\def\content{\uppercase{MC}}
\def\shortcontent{\content}
\def\versionnumber{1.2}  % Version of this reference card
\def\version{v\versionnumber\ \month\ \year}

% include stylefile
\input mydefs.sty
\mytitle

% shortcopyrightnotice
\shortintro

%%%%%%%%%%%%%%%%%%
%  card content  %
%%%%%%%%%%%%%%%%%%
\section{Commands}{}
\cmdS{\ctrl r}	{refresh active panel}
\cmdS{\ctrl c~\ctrl o}	{opens chmod/chown for marked file}
\cmdS{\alt ?}	{opens search dialog}
\cmdS{\ctrl a}	{open vfs list}
\cmdExmpl{}	{If a ftp session times out, you can use this to free the open vfs so you can log in again}

% section % commands (end)

\section{Panel Functions}{}
\cmdS{\tab}	{switch focus between left/right panel}
\cmdS{\key{insert}}	{marks or removes mark on files}
\cmdS{\alt g~\alt j}	{marks first/last file or dir in panel}
\cmdS{\alt r}	{marks middle file in active panel}
\cmdS{\alt s}	{incremental search}
\cmdS{$\ast$}	{marks \or removes marks for all files}
\cmdS{+~\bs}	{(un)mark files by regular expression}
\cmdS{\ctrl p~\ctrl n}	{moves up-/downwards in panel}

% section % panel_functions (end)

\section{Shell Functions}{}

\cmdS{\alt \enter}	{copies selected filename in command line}
\cmdS{\ctrl \shift \enter}	{copies full path of selected file to cliboard}
\cmdS{\alt h}	{command line history}

% section %  (end)

\section{Function Keys}{}
\cmdS{F1}	{Help}
\cmdS{F2}	{Opens user menu}
\cmdS{F3}	{View selected file content}
\cmdS{F4}	{Opens file in internal text editor}
\cmdS{F5}	{Copies selected file. Default is to another panel, but it asks first.}
\cmdS{F6}	{Moving file. Default is to another panel, but it asks first.}
\cmdS{F7}	{Make directory.}
\cmdS{F8}	{Delete file or directory.}
\cmdS{F9}	{Opens main menu at the top of the screen.}
\cmdS{F10}	{Ends current action; editor, viewer, dialog window or ends mc program.}

% section % function_keys (end)

\section{File Operations}{}
\cmdS{Enter}	{if  there  is  some text in the command line (the one at the
bottom of the panels), then that command is executed. If there is no text in
the command line then if the selection bar is over a directory the Midnight
Commander does a chdir(2) to the selected directory and reloads  the
information  on the panel; if the selection is an executable file then it is
executed. Finally, if the extension of the selected file name matches one of
the extensions in the extensions file then the corresponding command is
executed.}
\cmdS{\ctrl{l}}	{repaint all the information in the Midnight Commander.}
\cmdS{\ctrl{x} c}	{run the Chmod command on a file or on the tagged files.}
\cmdS{\ctrl{x} o}	{run the Chown command on the current file or on the tagged files.}
\cmdS{\ctrl{x} l}	{run the hard link command.}
\cmdS{\ctrl{x} s}	{run the absolute symbolic link command.}
\cmdS{\ctrl{x} v}	{run the relative symbolic link command. See the File Menu section for more information about symbolic links.}
\cmdS{\ctrl{x} i}	{set the other panel display mode to information.}
\cmdS{\ctrl{x} q}	{set the other panel display mode to quick view.}
\cmdS{\ctrl{x} !}	{execute the External panelize command.}
\cmdS{\ctrl{x} h}	{run the add directory to hotlist command.}
\cmdS{\alt !}	{executes the Filtered view command, described in the view command.}
\cmdS{\alt ?}	{executes the Find file command.}
\cmdS{\alt c}	{pops up the quick cd dialog.}
\cmdS{\ctrl{o}}	{when the program is being run in the Linux or FreeBSD console or under an xterm, it will show you the output of the previous command.   When  ran  on the Linux console, the Midnight Commander uses an external program (cons.saver) to handle saving and restoring of information on the screen.}

% section %  (end)

% Footer
\copyrightnotice

% Ending
\vfil
\supereject
\if L\lr \else\null\vfill\eject\fi
\if L\lr \else\null\vfill\eject\fi
\bye

% EOF
