% vim:fdm=marker:filetype=tex:ts=2:expandtab
% TMUX Quick Reference Card
% Copyright (c) 2014 Jonas Petong.

% version
\def\versionnumber{0.1}  % Version of this reference card
\def\year{2014}
\def\month{May}
\def\version{v\versionnumber\ \month\ \year}

% include definitions
\input mydefs.sty

% Card content
% Header
\title{TMUX QUICK REFERENCE CARD}

% \shortcopyrightnotice
\shortintro

\section{From Shell}{}

\cmdL{create session}	{tmux new -s {\tt session-name}}
\cmdL{kill session}	{tmux kill-session -t {\tt session-name}}
\cmdL{attach session}	{tmux attach -t {\tt session-name}}
\cmdL{rename session}	{tmux rename-session -t {\tt old-name} {\tt new-name}}
\cmdL{list session}	{tmux ls}

\section{Inside Tmux}{Prefix key is set to {\tt \ctrl b} by default. You can
    change that behaviour to ctrl-a by adding the following
    commands in your \tild/.tmux.conf:\\\\
: unbind C-b\\
: set -g prefix C-a\\
: bind C-a send-prefix\\
}

\subsection{Sessions}{}
\cmdS{\ctrl b?}	{show all available keybinds}
\cmdS{\ctrl bd \ctrl bs}	{detach / reattach to session}

\subsection{Copy Mode}{{\tt vim} like movements}
\cmdS{h j k l}{move left/down/up/right}
\cmdS{\ctrl b\lbracket ~ q}	{enter / exit copy mode}
\cmdS{\key{space} \enter}	{select / copy text under cursor}
\cmdS{\ctrl b\rbracket}	{paste copied text to command line}

\subsection{Windows}{}
\cmdS{\ctrl bc \ctrl b,}	{create / rename window}
\cmdS{\ctrl bn \ctrl bp \ctrl bl}	{navigate next/prev/last window}

\subsection{Panes/Buffers}{}
\cmdS{\ctrl b\%  \ctrl b"}	{split pane vertic / horiz}
\cmdS{\ctrl bx \or \ctrl d }	{delete pane}
\cmdS{\ctrl b$n$}	{change to buffer $0 \ldots n$}
\cmdS{\ctrl b\key{m - \arrows} }	{resize pane}
\cmdS{\ctrl b\key{\arrows} \ctrl bz}	{select / full-screen pane}

%% Footer
\copyrightnotice

% Ending
\supereject
\if L\lr \else\null\vfill\eject\fi
\if L\lr \else\null\vfill\eject\fi
\bye

% EOF
