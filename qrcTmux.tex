% vim:fdm=syntax:filetype=tex:ts=2:expandtab
% TMUX Quick Reference Card
% Copyright (c) 2014 Jonas Petong.

% version
\def\content{TMUX}
\def\versionnumber{1.0}  % Version of this reference card
\def\year{2014}
\def\month{May}
\def\version{v\versionnumber\ \month\ \year}

% include definitions
\input mydefs.sty

% Card content
% Header
\title{\content\ QUICK REFERENCE CARD}

% \shortcopyrightnotice
\shortintro\vskip 2ex

Prefix key is set to \key{ctrl-b} by default. You can change that
behaviour to e.g. \key{ctrl-a} by adding the following commands to your
\tild/.tmux.conf:\\

{\tt :unbind C-b\\ :set -g prefix C-a\\ :bind C-a send-prefix\\}

\section{From Shell}{}

\cmdL{create session}	{tmux new -s {\tt session-name}}
\cmdL{kill session}	{tmux kill-session -t {\tt session-name}}
\cmdL{attach session}	{tmux attach -t {\tt session-name}}
\cmdL{rename session}	{tmux rename-session -t {\tt old-name} {\tt new-name}}
\cmdL{list session}	{tmux ls}

\section{Inside Tmux}{\key{M} denotes the meta key, usually bound to \key{ALT}}

\subsection{Sessions}{}
\cmdS{\ctrl b?}	{show all available keybinds}
\cmdS{\ctrl bd \ctrl bs}	{detach / reattach to session}

\subsection{Copy Mode}{{\tt vim} like movements}
\cmdS{h j k l}{move left/down/up/right}
\cmdS{\ctrl b\lbracket ~ q}	{enter / exit copy mode}
\cmdS{\key{space} \enter}	{select / copy text under cursor}
\cmdS{\ctrl b\rbracket}	{paste copied text to command line}

\subsection{Windows}{}
\cmdS{\ctrl bc \ctrl b,}	{create / rename window}
\cmdS{\ctrl bf~\ctrl bw}	{find / list windows}
\cmdS{\ctrl bn \ctrl bp \ctrl bl}	{navigate next/prev/last window}

\subsection{Panes \or\ Buffers}{}
\subsubsection{Open / Switch / Resize}{}
\cmdS{\ctrl bx \or \ctrl d }	{delete pane}
\cmdS{\ctrl b!}	{close all other panes}
\cmdS{\ctrl bt \ctrl bq}	{show time / numeric values of panes}
\cmdS{\ctrl b\key{\arrows} \or\ctrl bo}	{switch to (next) pane}
\cmdS{\ctrl b$n$}	{switch to buffer $0 \ldots n$}
\cmdL{\ctrl b\key{M - \arrows } \ctrl bz}	{resize / full-screen pane}
\cmdS{\ctrl b$\rbrace$}	{swap position with next pane}
\cmdS{\ctrl b$\lbrace$ \or\ \ctrl b\ctrl o}	{swap with previous pane}

\subsubsection{Split Panes}{}
\cmdS{\ctrl b\%  \ctrl b"}	{split pane vertically / horizontally}
\cmdS{\ctrl b \key{M-1}}	{vertical split, all panes same width}
\cmdS{\ctrl b \key{M-2}}	{horizontal split, all panes same height}
\cmdS{\ctrl b \key{M-3}}	{horizontal split, main pane on top, other panes on bottom, vertically split, all same width}
\cmdS{\ctrl b \key{M-4}}	{vertical split, main pane left, other panes right, horizontally split, all same height}
\cmdS{\ctrl b \key{M-5}}	{tile, new panes on bottom, same height before same width}

%% Footer
\copyrightnotice

% Ending
\supereject
\if L\lr \else\null\vfill\eject\fi
\bye

% EOF
