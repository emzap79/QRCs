% TeX Format

% version
\def\content{VIMPERATOR}
\def\versionnumber{1.0}  % Version of this reference card


\def\version{v\versionnumber\ \month\ \year}

% include definitions
\input mydefs.sty

% Card content
% Header
%\hrule\vskip 3pt

\title{\uppercase{\content\ quick reference card}}

\shortintro

\vskip 0.5cm

% http://www.linuxjournal.com/article/10636
\section{Page \& Tab Movements}{}
\cmdS{hjkl}	{move left/down/up/right}
\cmdS{gg~G}	{move to top/bottom of page}
\cmdS{\ctrl f~\ctrl d~\ctrl j}	{move screen/page/line down}
\cmdS{\ctrl b~\ctrl u~\ctrl k}	{move screen/page/line up}
\cmdS{/~?}	{search in page (backwards)}
\cmdS{gt~\ctrl n}	{go to next tab}
\cmdS{gT~\ctrl p}	{go to previous tab}
\cmdS{g0~g\$}	{go to first/last tab}
\cmdS{gh~gH}	{go to your homepage (in new tab)}
\cmdS{go\lbrack aA-zZ0-9\rbrack}	{jump to previously marked web page}
\cmdS{gu~gU}	{move up one path/all to root}
\cmdExmpl{http://www.reddit.com/r/IAmA/comments moves up to www.redd\ldots /r/IAmA}	{gu}

\section{General Keybinds}{}
\cmdS{t~T}	{open tab (with smart completion)}
\cmdS{d~u}	{close/undo tab}
\cmdS{p~P}	{paste from clipboard (in a new tab)}
\cmdS{H~L}	{go to previous/next page in history}
\cmdS{a~A}	{add/toggle bookmark}
\cmdS{i~I}	{ignores all/specific VIMPRTR keybinds}
\cmdS{o~b \or .~}	{open page/buffer}
\cmdS{\lbracket\lbracket~\rbracket\rbracket}	{switch to previous/next page on multipage websites}
\cmdS{+~-}	{zoom in/out by 25 \% on page in focus}
\cmdS{$n$zz}	{zoom page to $n$\% (100 if $n$ is omitted)}
\cmdS{ZZ~ZQ}	{exit broweser and save/clear history}

\vfill

\section{Exe commands}{Works with tab completion}
\cmdS{:his\grey{tory}}	{pull up recent 10 online \& local pages}
\cmdS{:pc\grey{lose}}	{close preview window, eg. history}
\cmdL{:bm\grey{ark}~:delb\grey{marks} $url$}	{add/delete bookmark}
\cmdS{:res\grey{tart}}	{restart Firefox immediately}
\cmdS{:ad\grey{dons}}	{show addons in current window}
\cmdS{:pr\grey{eferences}}	{show preferences}

\section{Modes}{Hints are the way in which Vimperator allows you to follow links on a page. By
    providing each link with a suitable hint, you can access all links with a
        similar amount of minimal effort }
        \subsection{QuickHint Mode ($f$)}{Simulates mouseselection with labeled
            links on web page. To run new tab in background change to uppercase $F$.}
            \subsection{ExtendedHint Mode ( ; )}{This mode is useful for
                performing operations on hinted elements other than the default
                    left mouse click. If you want to yank the location of hint 24, press ;y to start this hint
                    mode. Then press 24 to copy the hint location.}
                    \cmdL{:h\grey{elp} extended-hints}	{Open list of all extended mode commands}
                    here are a few \ldots\\
                        \cmdS{a~A $n$}	{open link $n$ (in new background tab)}
                        \cmdS{y~Y}	{copy URL/label description to clipboard}
                        \cmdS{s~S}	{save destination of a link}

% Footer
\copyrightnotice

% Ending
\vfil
\supereject
\if L\lr \else\null\vfill\eject\fi
\if L\lr \else\null\vfill\eject\fi
\bye

% EOF
