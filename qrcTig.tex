% vim:ts=2:expandtab
% TIG Quick Reference Card
% Copyright (c) 2011 Gabriel Burca.
% TeX Format - Based on VIM Quick Reference Card

% compile as 'pdftex tig-qrc.tex'

\def\versionnumber{1.0}  % Version of this reference card
\def\year{2013}
\def\month{Oct}
\def\version{v\versionnumber\ \month\ \year}

\def\shortcopyrightnotice{\smallskip
  \centerline{\smallrm \copyright\ \year\ Gabriel Burca -
  Permissions on back. \version}}

\def\copyrightnotice{
\vfill \hrule\smallskip\begingroup\smallrm
\centerline{\version. Copyright \copyright\ \year\ Gabriel B. Burca}

Permission is granted to make and distribute copies of this card provided
the copyright notice and this permission notice are preserved on all
copies. Send comments or corrections to Gabriel Burca at:
$\langle$gburca-GitQRC@ebixio.com$\rangle$

\centerline{http://ebixio.com/ and http://github.com/gburca/git-qrc}
\endgroup}

% \pdfoutput=1
\pdfpageheight=21cm
\pdfpagewidth=29.7cm

% Font definitions
\font\bigbf=cmbx12
\font\smallrm=cmr8
\font\smalltt=cmtt8
\font\tinyit=cmmi5

\def\title#1{\hfil{\bf #1}\hfil\par\vskip 2pt\hrule}
\def\section#1{\vskip 0.7cm {\it#1\/}\par\vskip 1pt\hrule}


% Characters definitions
\def\\{\hfil\break}
\def\bs{$\backslash$}
\def\backspace{$\leftarrow$}
\def\ctrl{{\rm\char94}\kern-1pt}
\def\enter{$\hookleftarrow$}
\def\or{\thinspace{\tinyit{or}}\thinspace}
\def\key#1{$\langle${\rm{\it#1\/}}$\rangle$}
\def\rapos{\char125}
\def\lapos{\char123}
\def\bs{\char92}
\def\tild{\char126}
\def\lbracket{[}
\def\rbracket{]}

% Three columns definitions
\parindent 0pt
\nopagenumbers
\hoffset=-1.56cm
\voffset=-1.54cm
\newdimen\fullhsize
\fullhsize=27.9cm
\hsize=8.5cm
\vsize=19cm
\def\fullline{\hbox to\fullhsize}
\let\lr=L
\newbox\leftcolumn
\newbox\midcolumn
\output={
  \if L\lr
    \global\setbox\leftcolumn=\columnbox
    \global\let\lr=M
  \else\if M\lr
    \global\setbox\midcolumn=\columnbox
    \global\let\lr=R
  \else
    \tripleformat
    \global\let\lr=L
  \fi\fi
  \ifnum\outputpenalty>-20000
  \else
    \dosupereject
  \fi}
\def\tripleformat{
  \shipout\vbox{\fullline{\box\leftcolumn\hfil\box\midcolumn\hfil\columnbox}}
  \advancepageno}
\def\columnbox{\leftline{\pagebody}}

% GIT optimizations
\def\cmd#1{{\tt#1}\null}	%\null so not an abbrev even if period follows

% A short command
\def\cmdS#1#2{
  \noindent
  \hbox to \hsize {% This comment is needed to remove space at start of line
    \vtop{
      \hbox to 2.4cm {
        \noindent\cmd{#1}\dotfill
      }
      %\hrule
    }% This comment is needed to remove space
    \hfil	% absorb extra space b/w columns
    \vtop{
      \hsize=5.90cm
      \hbox{\hfuzz = 15pt \vtop{
      % Uncomment line below to right-align description
      %\leftskip=0cm plus 150fil
      {#2}
      }}
      %\hrule
    }
  }
  %\hrule
  \par
  \vskip 0.14cm
}

% A long command
\def\cmdL#1#2{
  \hsize=8.5cm
  \vbox {
    \hbox{
        \cmd{#1}\hfil
    }%
    %\vskip -0.2cm % adjusts space b/w command & description
    \hskip 2.5cm  % indent for description
    \hbox to 5.9cm {%
      \hfuzz = 5pt
      % Use the first one when right-aligned
      %\vrule\hskip -0.05cm
      %\vrule\hskip 0.05cm
      \hfil
      \hsize=5.9cm
      \vtop{
        %\hrule
        % Uncomment line below to right-align description
        %\leftskip=0cm plus 150fil
        {#2}
        }}
  }%
  %\hrule
  \par
  \vskip 0.14cm
}


% Card content
% Header
\title{TIG QUICK REFERENCE CARD}

\shortcopyrightnotice

\section{Views}

\cmdS{main} {
This is the default view, and it shows a one line summary of each commit in the
chosen list of revisions. The summary includes commit date, author, and the
first line of the log message. Additionally, any repository references, such as
tags, will be shown.}

\cmdS{log} {
Presents a more rich view of the revision log showing the whole log message and
the diffstat.}

\cmdS{diff} {
Shows either the diff of the current working tree, that is, what has changed
since the last commit, or the commit diff complete with log message, diffstat
and diff.}

\cmdS{tree} {
Lists directory trees associated with the current revision allowing
subdirectories to be descended or ascended and file blobs to be viewed.}

\cmdS{blob} {
Displays the file content or "blob" of data associated with a file name.}

\cmdS{blame} {
Displays the file content annotated or blamed by commits.}

\cmdS{branch} {
Displays the branches in the repository.}

\cmdS{status} {
Displays status of files in the working tree and allows changes to be staged /
unstaged as well as adding of untracked files.}

\cmdS{stage} {
Displays diff changes for staged or unstanged files being tracked or file
content of untracked files.}

\cmdS{pager} {
Is used for displaying both input from stdin and output from git commands
entered in the internal prompt.}

\cmdS{help} {
Displays a quick reference of key bindings.}


\vskip 1cm
\section{View Switching Keys}
\cmdS{m} {Switch to main view.}

\cmdS{d} {Switch to diff view.}

\cmdS{l} {Switch to log view.}

\cmdS{p} {Switch to pager view.}

\cmdS{t} {Switch to (directory) tree view.}

\cmdS{f} {Switch to (file) blob view.}

\cmdS{B} {Switch to blame view.}

\cmdS{H} {Switch to branch view.}

\cmdS{h} {Switch to help view.}

\cmdS{S} {Switch to status view.}

\cmdS{c} {Switch to stage view.}


\section{View Manipulation Keys}
\cmdS{q} {Close view, if multiple views are open it
               will jump back to the previous view in the
               view stack. If it is the last open view it
               will quit. Use Q to quit all views at once.}

\cmdS{\key{Enter}} {This key is "context sensitive" depending on
               what view you are currently in. When in log
               view on a commit line or in the main view,
               split the view and show the commit diff. In
               the diff view pressing Enter will simply
               scroll the view one line down.}

\cmdS{\key{Tab}} {Switch to next view.}

\cmdS{R} {Reload and refresh the current view.}

\cmdS{M} {Maximize the current view to fill the whole
               display.}

\cmdS{\key{Up}} {This key is "context sensitive" and will
               move the cursor one line up. However, if you
               opened a diff view from the main view
               (split- or full-screen) it will change the
               cursor to point to the previous commit in
               the main view and update the diff view to
               display it.}

\cmdS{\key{Down}} {Similar to Up but will move down.}

\cmdS{,} {Move to parent. In the tree view, this means
               switch to the parent directory. In the blame
               view it will load blame for the parent
               commit. For merges the parent is queried.}


\section{View Specific Keys}
\cmdS{u} {Update status of file. In the status view,
             this allows you to add an untracked file or
             stage changes to a file for next commit
             (similar to running git-add \key{filename}). In
             the stage view, when pressing this on a diff
             chunk line stages only that chunk for next
             commit, when not on a diff chunk line all
             changes in the displayed diff is staged.}

\cmdS{M} {Resolve unmerged file by launching
             git-mergetool(1). Note, to work correctly
             this might require some initial
             configuration of your preferred merge tool.
             See the manpage of git-mergetool(1).}

\cmdS{!} {Checkout file with unstaged changes. This
             will reset the file to contain the content
             it had at last commit.}

\cmdS{@} {Move to next chunk in the stage view.}


\section{Cursor Navigation Keys}
\cmdS{k} {Move cursor one line up.}

\cmdS{j} {Move cursor one line down.}

\cmdS{\key{PgUp}\or -\or a} {Move cursor one page up.}

\cmdL{\key{PgDown}\or \key{Space}} {Move cursor one page down.}

\cmdS{\key{End}} {Jump to last line.}

\cmdS{\key{Home}} {Jump to first line.}


\vskip 2.5cm
\section{Scrolling Keys}
\cmdS{\key{Insert}} {Scroll view one line up.}

\cmdS{\key{Delete}} {Scroll view one line down.}

\cmdS{w} {Scroll view one page up.}

\cmdS{s} {Scroll view one page down.}

\cmdS{\key{Left}} {Scroll view one column left.}

\cmdS{\key{Right}} {Scroll view one column right.}

\cmdS{|} {Scroll view to the first column.}


\section{Searching Keys}
\cmdS{/} {Search the view. Opens a prompt for entering
             search regexp to use.}

\cmdS{?} {Search backwards in the view. Also prompts for regexp.}

\cmdS{n} {Find next match for the current search regexp.}

\cmdS{N} {Find previous match for the current search regexp.}

\section{Misc Keys}
\cmdS{Q} {Quit.}

\cmdS{r} {Redraw screen.}

\cmdS{z} {Stop all background loading. This can be useful if you use tig in a
repository with a long history without limiting the revision log.}

\cmdS{v} {Show version.}

\cmdS{o} {Open option menu}

\cmdS{.} {Toggle line numbers on/off.}

\cmdS{D} {Toggle date display on/off/short/\\relative/local.}

\cmdS{A} {Toggle author display on/off/abbreviated.}

\cmdS{g} {Toggle revision graph visualization on/off.}

\cmdS{~} {Toggle (line) graphics mode}

\cmdS{F} {Toggle reference display on/off (tag and
             branch names).}

\cmdS{:} {Open prompt. This allows you to specify what
             git command to run. Example :log -p. You can
             also use this to jump to a specific line by
             typing :$<$linenumber$>$, e.g. :80.}

\cmdS{e} {Open file in editor.}


%% Footer
\copyrightnotice

% Ending
\supereject
\if L\lr \else\null\vfill\eject\fi
\if L\lr \else\null\vfill\eject\fi
\bye

% EOF
